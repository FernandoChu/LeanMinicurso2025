\documentclass{beamer}
\usepackage{mathpartir}
\usepackage[style=authoryear]{biblatex}
\usepackage{stmaryrd}
\usepackage{amsmath,amssymb,amsfonts,amsthm}
\usepackage{xcolor}
\usepackage{multicol}
\usepackage{quiver}
\usepackage{graphicx}
\graphicspath{ {./images/} }

\addbibresource{biblio.bib}
\setbeamercolor{bibliography entry author}{fg=blue}
\setbeamercolor{bibliography entry location}{fg=lightgray}
\setbeamercolor{bibliography entry note}{fg=lightgray}

\newcommand{\coerce}{\mathsf{coerce}}
% \newcommand{\mode}{\mathbb M}
\newcommand{\id}{\mathsf{id}}
\newcommand{\UU}{\mathcal{U}}
\newcommand{\op}{\mathsf{op}}
\newcommand{\core}{\mathsf{core}}
\newcommand{\loc}{\mathsf{loc}}
\newcommand{\Cat}{\mathsf{Cat}}
\newcommand{\Set}{\mathsf{Set}}
\newcommand{\C}{\mathcal C}
\newcommand{\gcs}[1]{\overset{\scriptscriptstyle #1}{.}}
\newcommand{\isovar}{\overset{\circ}{:}}
\newcommand{\covar}{\overset{\scriptscriptstyle  +}{:}}
\newcommand{\contravar}{\overset{\scriptscriptstyle -}{:}}
\newcommand{\nvar}{\overset{\scriptscriptstyle \times}{:}}
\newcommand{\var}[1]{\overset{\scriptscriptstyle #1}{:}}
\newcommand{\ctx}{\,\mathsf{ctx}}
\newcommand{\gctx}{\,\mathsf{gctx}}
\newcommand{\type}{\ \mathsf{type}}
\newcommand{\ort}{\ \mathsf{ort}}
\newcommand{\xort}[1]{\ \operatorname{#1-\mathsf{ort}}}
\newcommand{\x}{\mathnormal{x}}
\newcommand{\Id}{\mathsf{Id}}
\newcommand{\refl}{\mathsf{refl}}
\newcommand{\Tm}{\mathsf{Tm}}
\newcommand{\cextp}[2]{#1.#2}
\newcommand{\cextn}[2]{#1.#2}
\newcommand{\ind}{\mathsf{ind}}
\newcommand{\flip}{\mathsf{flip}\,}
\newcommand{\unflip}{\mathsf{unflip}\,}
\newcommand{\strip}{\mathsf{strip}\,}
\newcommand{\unstrip}{\mathsf{strip}\,}
\newcommand{\homset}{\mathsf{hom}\mathcal{S}\mathsf{et}}
\newcommand{\defeq}{:\equiv}
\newcommand{\tr}{\mathsf{tr}}
\newcommand{\Ty}{\mathsf{Ty}}
\newcommand{\Con}{\mathsf{Con}}
\newcommand{\Ort}{\mathsf{Ort}}
\newcommand{\OrtType}{\mathsf{OrtType}}
\newcommand{\List}{\mathsf{List}}
\newcommand{\dfunext}{\mathsf{funext}^\to}
\newcommand{\funext}{\mathsf{funext}^=}
\newcommand{\ext}[1]{\overset{\scriptscriptstyle #1}{.}}

\usetikzlibrary{calc,decorations.pathmorphing,shapes}
\newcounter{sarrow}
\newcommand\xrsquigarrow[1]{%
  \stepcounter{sarrow}%
  \mathrel{
    \begin{tikzpicture}[baseline= {( $ (current bounding box.south) + (0,-0.5ex) $ )}]
      \node[inner sep=.5ex] (\thesarrow) {$\scriptstyle #1$};
      \path[draw,<-,decorate,
      decoration={zigzag,amplitude=0.7pt,segment length=1.2mm,pre=lineto,pre length=4pt}]
      (\thesarrow.south east) -- (\thesarrow.south west);
  \end{tikzpicture}}%
}

\useoutertheme[subsection = false, compress, footline = false]{miniframes}

\usefonttheme[onlymath]{serif}

\definecolor{red}      {HTML}{ee9999}
\definecolor{green}    {HTML}{aaddaa}
\definecolor{blue}     {HTML}{9999ee}
\definecolor{yellow}   {HTML}{ddddaa}
\definecolor{pink}     {HTML}{ddaadd}
\definecolor{mizu}     {HTML}{aadddd}
\definecolor{gray}     {HTML}{333333}
\definecolor{lightgray}{HTML}{dddddd}
\definecolor{darkgray} {HTML}{222222}

\setbeamercolor{palette primary}     {fg = white, bg = gray}
\setbeamercolor{palette secondary}   {fg = white, bg = lightgray}
\setbeamercolor{palette tertiary}    {fg = white, bg = gray}
\setbeamercolor{background canvas}   {fg = white, bg = gray}
\setbeamercolor{title}               {fg = green, bg = gray}
\setbeamercolor{frametitle}          {fg = green, bg = gray}
\setbeamercolor{normal text}         {fg = white, bg = gray}
\setbeamercolor{alerted text}        {fg = green, bg = gray}
\setbeamercolor{block title}         {fg = gray,  bg = lightgray}
\setbeamercolor{block body}          {fg = white, bg = darkgray}
\setbeamercolor{block title example} {fg = gray,  bg = green}
\setbeamercolor{block body  example} {fg = white, bg = green!33!black}
\setbeamercolor{block title alerted} {fg = gray,  bg = red}
\setbeamercolor{block body  alerted} {fg = white, bg = red!33!black}
\setbeamercolor{qed symbol}          {fg = white, bg = gray}
\setbeamercolor{enumerate item}      {fg = green}
\setbeamercolor{enumerate subitem}   {fg = green}
\setbeamercolor{enumerate subsubitem}{fg = green}
\setbeamercolor{itemize   item}      {fg = green}
\setbeamercolor{itemize   subitem}   {fg = green}
\setbeamercolor{itemize   subsubitem}{fg = green}

\setbeamerfont{title}      {series = \bfseries, size = \LARGE}
\setbeamerfont{frametitle} {series = \bfseries, size = \Large}
\setbeamerfont{block title}{series = \bfseries}

\beamertemplatenavigationsymbolsempty
\setbeamercovered{transparent = 12.5}

\setbeamertemplate{title page}{
  \centering
  \vbox{}
  \vskip3em
  \usebeamertemplate{title}
  \usebeamertemplate{author}
  \usebeamertemplate{date}
}

\newenvironment{smallblock}
{
\begin{block}{\vspace{-10pt}}}
  {
\end{block}}

% Select some frames to appear in the handout version
\makeatletter
\newif\ifOnBeamerModeTransition
\newcommand{\slideselection}{1-}%
\newenvironment{handoutframeselect}[1][1-]{%
  \begingroup%
  \mode<handout>{%
    \gdef\beamer@currentmode{beamer}%
    \OnBeamerModeTransitiontrue%
  \renewcommand{\slideselection}{#1}}%
}{%
  \ifOnBeamerModeTransition%
  \OnBeamerModeTransitionfalse%
  \gdef\beamer@currentmode{handout}%
  \fi%
  \endgroup%
}
\makeatother

% Don't index page
\makeatletter
\let\beamer@writeslidentry@miniframeson=\beamer@writeslidentry%
\def\beamer@writeslidentry@miniframesoff{\expandafter\beamer@ifempty\expandafter{\beamer@framestartpage}{}{\clearpage\beamer@notesactions}}
\newcommand*\miniframeson{\let\beamer@writeslidentry=\beamer@writeslidentry@miniframeson}
\newcommand*\miniframesoff{\let\beamer@writeslidentry=\beamer@writeslidentry@miniframesoff}
\makeatother

\setbeamertemplate{itemize item}{\textbullet}
\setbeamertemplate{itemize subitem}{\textopenbullet}

% \setbeamercolor{block title}         {fg = gray,  bg = lightgray}
% \setbeamercolor{block body}          {fg = white, bg = darkgray}

\newtheorem{proposition}{Proposition}
\newtheorem{oldproposition}{Proposition}
\AtBeginEnvironment{oldproposition}{%
  \setbeamercolor{block title}{use=example text,fg=gray,bg=lightgray!75!white}
  \setbeamercolor{block body}{parent=normal text,use=block title example,bg=gray!75!white}
  % \setbeamercolor{block body}{parent=normal text,use=block title example,bg=gray!10}
}

\title[]{Formalizando Matem\'aticas en Lean}

\author{Fernando Chu}

\date{Diciembre 2025 - IMCA}

\begin{document}
\frame{\titlepage}

\begin{frame}[plain]
  \centering
  \Huge \alert{?`Qu\'e es Lean?}
\end{frame}

\begin{frame}
  \frametitle{?`Qu\'e es Lean?}
  \Large
  \only<1>{
    Lean es un lenguaje de programaci\'on funcional y un asistente de pruebas de c\'odigo abierto.
  }
  \only<2>{
    Lean es un lenguaje de programaci\'on funcional y un asistente de pruebas de \alert{c\'odigo abierto}.
  }
  \only<3>{
    Lean es un \alert{lenguaje de programaci\'on funcional} y un asistente de pruebas de c\'odigo abierto.
  }
  % Mathematica, matlab son closed source
  \only<4>{
    Lean es un lenguaje de programaci\'on funcional y un \alert{asistente de pruebas} de c\'odigo abierto.
  }
\end{frame}

\begin{frame}[plain]
  \centering
  \Huge \alert{?`Por qu\'e Lean? \\ ?`Por qu\'e formalizar?}
\end{frame}

\begin{frame}
  \frametitle{Certeza de los resultados}
  Recordatorio: Todos cometemos errores!
  \\[1em]

  Algunos ejemplos:
  \begin{itemize}
      \pause
    \item La primera prueba del Teorema de Fermat de Wiles (1993) estaba equivocada.
      \pause
    \item La hip\'otesis de homotop\'ia fue demostrada por Kapranov y Voevodsky (1991). A\~nos despu\'es, Simpson encontr\'o un contraejemplo (1998)!
      \pause
    \item Scholze, escribe sobre su propio teorema (2020):
      \begin{quote}
        ...I think the theorem is of utmost foundational importance, so being 99.9\% sure is not enough.
      \end{quote}
  \end{itemize}
\end{frame}

\begin{frame}
  \frametitle{Entendimiento de los resultados}
  Scholze ret\'o a la comunidad de Lean a que formalizara su teorema (2020), seis meses despu\'es el reto fue cumplido.
  \'El escribe:
  \\[1em]
  \begin{quotation}
    When I wrote the blog post half a year ago, I did not understand why the argument worked, (\dots) during the formalization, a significant amount of convex geometry had to be formalized (\dots) this made me realize that actually the key thing happening is a reduction from a non-convex problem over the reals to a convex problem over the integers.
  \end{quotation}
\end{frame}

\begin{frame}
  \frametitle{Otras razones}
  \begin{itemize}
    \item Asistencia en las pruebas: \\ Ej., casos m\'ultiples, argumentos rutinarios.
      \vspace{1em}
    \item Asistencia en la ense\~nanza: \\ Ej., Tao formaliz\'o su libro de an\'alisis, hay cursos de primer a\~no en varias universidades.
      \vspace{1em}
    \item Colaboraci\'on de las matem\'aticas: \\ Ej., paralelizaci\'on del trabajo: ``Equational Theories Project'' (Tao, 2024) con 65 contribuidores! Buzzard y FLT, etc.
  \end{itemize}
\end{frame}

\begin{frame}
  \frametitle{Otras razones}
  \begin{itemize}
    \item Incentiva buenas abstracciones: \\ Ej., Johan Commelin cre\'o una nueva estructura algebraica para demostrar el teorema de Scholze.
      \vspace{1em}
    \item Archivo abierto de las matem\'aticas: \\ Definiciones y pruebas ``del Libro''. Preservaci\'on de resultados a trav\'es de generaciones.
      \vspace{1em}
    \item Posiciones laborales: \\ Posiciones en academia e industria.
      \vspace{1em}
    \item (Especulativo) Un requisito en un futuro: \\Grandes matem\'aticos ya se est\'an adelantando, y formalizando sus resultados (Tao, Riehl, Scholze, etc).
  \end{itemize}
\end{frame}

\begin{frame}[plain]
  \centering
  \Huge
  \alert{Una raz\'on m\'as:}\\
  !`Es divertido!
\end{frame}

\begin{frame}[plain]
  \centering
  \Huge \alert{?`Preguntas?}
\end{frame}

\begin{frame}
  \frametitle{Mathlib}
  No queremos definir todas las matem\'aticas desde 0.
  \pause
  \vspace{2em}

  Queremos una librer\'ia con todas las matem\'aticas b\'asicas:

  \centering
  \pause
  \Huge \alert{!`Mathlib!}
\end{frame}

\begin{frame}
  \frametitle{Mathlib}
  Mathlib es la librer\'ia de matem\'aticas formalizadas m\'as grande del mundo.
  \begin{itemize}
    \item Desarrollada por m\'as de 100 contribuyentes
    \item M\'as de 2 millones de l\'ineas de c\'odigo.
    \item B\'asicamente toda la matem\'atica de nivel de pregrado ya est\'a formalizada.
    \item Necesitamos m\'as contribuyentes!
  \end{itemize}

  \vspace{2em}
  Los invito a unirse a la comunidad de Lean en Zulip: \url{https://leanprover.zulipchat.com/}
\end{frame}

\begin{frame}
  \frametitle{Plan (Tentativo)}
  \begin{enumerate}
    \item D\'ia 1:
      \begin{itemize}
        \item L\'ogica y T\'acticas b\'asicas
        \item Sucesiones y T\'acticas adicionales
      \end{itemize}
    \item D\'ia 2:
      \begin{itemize}
        \item \'Algebra, estructuras y clases
        \item Derivadas y coerciones
      \end{itemize}
    \item D\'ia 3:
      \begin{itemize}
        \item Tipos inductivos
        \item Enteros, racionales y cocientes
      \end{itemize}
    \item D\'ia 4:
      \begin{itemize}
        \item Otros comentarios sobre formalizaci\'on
        \item Mini proyecto
      \end{itemize}
  \end{enumerate}
\end{frame}

\begin{frame}[plain]
  \centering
  \Huge \alert{Formalizemos!} \\[3em]
  \large
  \url{https://tinyurl.com/leanimca}
  \url{https://github.com/FernandoChu/LeanMinicurso2025}

\end{frame}

\end{document}